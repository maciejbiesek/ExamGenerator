\documentclass[addpoints]{exam}
\usepackage{polski}
\usepackage[utf8]{inputenc}
\pagestyle{headandfoot}
\firstpageheader{\large\bfseries NAZWA PRZEDMIOTU SOP}{}{\large\bfseries GRUPA 2}

\runningfooter{}{Strona \thepage}{}
\runningfooter{}{Page \thepage\ of \numpages}{}

\begin{document}
\raggedleft \textbf {Data:\enspace\makebox[2in]{\hrulefill}} 
\hfill  \textbf{Imię i nazwisko: \enspace\makebox[2in]{\hrulefill}}
	
\vspace{0.1in}

	\begin{questions}

	 %\tags{newton}
  \question
    Korzystając z~metody stycznych wyznacz wartość pierwiastka z~dwóch
    z~dokładnością do jednej dziesięciotysiącznej.

 %\tags{dokladnosc,odsetki-skladane}
  \question
    Na konto oprocentowane w~wysokości 0,45\% z~kapitalizacją
    miesięczną wpłacono 100zl.  Jakie będzie saldo konta po roku?
    Zastosuj do obliczeń wzór $K_n=K_0(1+\frac{p}{k})^{kn}$,
    a~następnie wylicz stan konta po każdym miesiącu, zaokrąglając go
    każdorazowo do pełnych groszy.  Porównaj wyniki.

 %\tags{odsetki-skladane}
  \question
    Pewien kapitał złożono na 2\% w~skali roku.  Po jakim czasie saldo
    rachunku przekroczy dwukrotność stanu początkowego?

 %\tags{odsetki-skladane,rozne-okresy-kapitalizacji}
  \question
    Kapitał $12\,000zł$ jest oprocentowany stopą 5\% rocznie przy
    kapitalizacji kwartalnej.  Wyznacz odsetki należne za ósmy kwartał.

 %\tags{odsetki-skladane,rozne-okresy-kapitalizacji,kapitalizacja-z-gory,kapitalizacja-ciagla}
  \question
    Do jakiej kwoty wzrośnie w~ciągu dwóch lat kapitał początkowy $600zł$ przy
    stopie procentowej 3\% i~kapitalizacji:
    \begin{enumerate}
    \item rocznej z~dołu;
    \item miesięcznej z~dołu;
    \item dziennej z~dołu;
    \item ciągłej;
    \item dziennej z~góry;
    \item miesięcznej z~góry;
    \item rocznej z~góry?
    \end{enumerate}



	\end{questions}
\end{document}

data:\enspace\makebox[2in]{\hrulefill}}
