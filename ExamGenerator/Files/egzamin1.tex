\documentclass[addpoints]{exam}
\usepackage{polski}
\usepackage[utf8]{inputenc}
\pagestyle{headandfoot}
\firstpageheader{\large\bfseries NAZWA PRZEDMIOTU ALGORYTMY TRZY}{}{\large\bfseries GRUPA 0}

\runningfooter{}{Strona \thepage}{}
\runningfooter{}{Page \thepage\ of \numpages}{}

\begin{document}
\raggedleft \textbf {Data:\enspace\makebox[2in]{\hrulefill}} 
\hfill  \textbf{Imię i nazwisko: \enspace\makebox[2in]{\hrulefill}}
	
\vspace{0.1in}

	\begin{questions}

	 %\tags{odsetki-skladane,kapitalizacja-z-gory}
  \question
    Po ilu latach kapitał złożony na 4\% w~kapitalizacji z~góry podwoi się?

 %\tags{odsetki-skladane,modelowanie}
  \question
    Pewien bank zaoferował swoim klientom roczną lokatę
    tzw. progresywną, tj. o~rosnącym oprocentowaniu.  Środki na tej
    lokacie są kapitalizowane na koniec każdego miesiąca stopą
    procentową wynoszącą 4\% w~pierwszym miesiącu i~rosnącą o~1~punkt
    procentowy w~każdym kolejnym miesiącu.
    \begin{enumerate}
    \item Jakie jest efektywne roczne oprocentowanie tej lokaty?
    \item Jaka byłaby odpowiedź, gdyby kapitalizacja następowała co
      kwartał (przy tych samych stopach procentowych za kolejne
      miesiące)?
    \end{enumerate}

 %\tags{odsetki-proste,odsetki-skladane,kapitalizacja-ciagla,kapitalizacja-z-gory}
  \question
    Stopa procentowa wynosiła 4\% w~pierwszym roku, a~przez kolejne dwa lata
    spadała o~pół punktu procentowego rocznie.  Wyznacz wartość kapitału
    $800zł$ za 3~lata, jeśli odsetki były:
    \begin{enumerate}
    \item proste;
    \item składane z~dołu;
    \item naliczane w~sposób ciągły;
    \item składane z~góry co kwartał.
    \end{enumerate}



	\end{questions}
\end{document}

data:\enspace\makebox[2in]{\hrulefill}}
