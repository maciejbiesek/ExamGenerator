\documentclass[addpoints]{exam}
\usepackage{polski}
\usepackage[utf8]{inputenc}
\pagestyle{headandfoot}
\firstpageheader{\large\bfseries NAZWA PRZEDMIOTU AFP}{}{\large\bfseries GRUPA 0}

\runningfooter{}{Strona \thepage}{}
\runningfooter{}{Page \thepage\ of \numpages}{}

\begin{document}
\raggedleft \textbf {Data:\enspace\makebox[2in]{\hrulefill}} 
\hfill  \textbf{ImiÄ™ i nazwisko: \enspace\makebox[2in]{\hrulefill}}
	
\vspace{0.1in}

	\begin{questions}

	 %\tags{modelowanie,odsetki-skladane,rozne-okresy-kapitalizacji,dokladnosc}
  \question
    Bank oferuje lokatę 15-dniową ,,Achilles'' oprocentowaną na 8,6\%
    p.a.  Po okresie lokaty kapitał wraz z~odsetkami przelewany jest
    na rachunek bieżący ,,Żółw'', oprocentowany na 1,2\% w~skali
    roku.  Jaka jest faktyczna opłacalność długookresowego lokowania
    pieniędzy w~tym banku, jeśli lokatę ,,Achilles'' można zakładać
    wyłącznie we wtorki?  (Przyjąć, że każdy rok ma 365 dni.)

 %\tags{odsetki-skladane,rozne-okresy-kapitalizacji,kapitalizacja-z-gory,kapitalizacja-ciagla}
  \question
    Do jakiej kwoty wzrośnie w~ciągu dwóch lat kapitał początkowy $600zł$ przy
    stopie procentowej 3\% i~kapitalizacji:
    \begin{enumerate}
    \item rocznej z~dołu;
    \item miesięcznej z~dołu;
    \item dziennej z~dołu;
    \item ciągłej;
    \item dziennej z~góry;
    \item miesięcznej z~góry;
    \item rocznej z~góry?
    \end{enumerate}

 %\tags{odsetki-skladane,kapitalizacja-z-gory,kapitalizacja-ciagla}
  \question
    Wyznacz odsetki naliczone w~ciągu roku od kwoty $500zł$, jeśli stopa
    procentowa wynosi 4\%, a~kapitalizacja jest: (a)~kwartalna z~góry;
    (b)~ciągła.

 %\tags{odsetki-skladane,kapitalizacja-z-gory}
  \question
    Jaka musi być stopa procentowa, by odsetki od $100zł$ wyniosły $10zł$ po
    roku kapitalizacji z~góry?



	\end{questions}
\end{document}

data:\enspace\makebox[2in]{\hrulefill}}
