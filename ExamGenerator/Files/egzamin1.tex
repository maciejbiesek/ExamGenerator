\documentclass[addpoints]{exam}
\usepackage{polski}
\usepackage[utf8]{inputenc}
\pagestyle{headandfoot}
\firstpageheader{\large\bfseries NAZWA PRZEDMIOTU}{}{\large\bfseries grupa GRUPA}

\runningfooter{}{Strona \thepage}{}
\runningfooter{}{Page \thepage\ of \numpages}{}

\begin{document}
\raggedleft \textbf {Data:\enspace\makebox[2in]{\hrulefill}} 
\hfill  \textbf{ImiÄ™ i nazwisko: \enspace\makebox[2in]{\hrulefill}}
	
\vspace{0.1in}

	\begin{questions}

	Znajdź zbiór tych wartości parametru m dla których równanie m·2x + (m + 3) ·2-x - 4 = 0 ma co najmniej jedno rozwiązanie.
Dane są punkty A = (2,1), B = (4,1), S1 = (- 22,1 ) i S 2 = (8,1) . Odcinek CD jest obrazem odcinka AB w jednokładności o skali dodatniej i środku S 1 , jak i w jednokładności o skali ujemnej i środku S2 . Oblicz współrzędne punktów C i D.
Wykaż, że log7 5 = log49 25 .


	\end{questions}
\end{document}

data:\enspace\makebox[2in]{\hrulefill}}
