\documentclass[addpoints]{exam}
\usepackage{polski}
\usepackage[utf8]{inputenc}
\pagestyle{headandfoot}
\firstpageheader{\large\bfseries NAZWA PRZEDMIOTU}{}{\large\bfseries grupa GRUPA}

\runningfooter{}{Strona \thepage}{}
\runningfooter{}{Page \thepage\ of \numpages}{}

\begin{document}
\raggedleft \textbf {Data:\enspace\makebox[2in]{\hrulefill}} 
\hfill  \textbf{ImiÄ™ i nazwisko: \enspace\makebox[2in]{\hrulefill}}
	
\vspace{0.1in}

	\begin{questions}

	 %\tags{procenty}
  \question
    Udowodnij, że jeśli pewną kwotę najpierw zwiększymy o~pewien
    procent, a~następnie zmniejszymy o~ten sam procent, to zawsze
    otrzymamy kwotę niższą od wyjściowej.  Jakie założenia są tu
    potrzebne?

 %\tags{procenty}
  \question
    Dwie firmy miały w~ofercie pewną usługę w~tej samej cenie.
    Pierwsza firma podwyższyła cenę usługi o~$p_1$ procent, a~druga
    obniżyła ją o~$p_2$~procent; następnie pierwsza firma obniżyła
    cenę o~$p_2$ procent, a~druga podwyższyła cenę o~$p_1$~procent.
    W~której firmie usługa jest teraz tańsza?

 %\tags{procenty}
  \question
    Pewien towar kosztował $100zł$.  Ile wynosi jego cena po podwyżce
    o~$10\%$, a~następnie obniżce o~$10\%$?

 %\tags{procenty,modelowanie}
  \question
    Koszt wytworzenia pewnego produktu wynosi $53zł$.  Jaka musi być
    cena produktu, by przy pięcioprocentowym rabacie producent uzyskał
    $15\%$ marży?

 %\tags{procenty,modelowanie}
  \question
    Dwóch handlarzy zaopatruje się w~pewien towar u~hurtownika.  Cena
    hurtowa wynosi $16zł$ za sztukę.  Obaj stosują upusty dla stałych
    klientów: pierwszy~$4\%$, zaś drugi~$16\%$.  Pierwszy handlarz
    planuje marżę~$16\%$ (przy sprzedaży stałemu klientowi), zaś
    drugi~$4\%$.  U~którego z~nich stały klient uzyska niższą cenę?



	\end{questions}
\end{document}

data:\enspace\makebox[2in]{\hrulefill}}
